\chapter{Trabalho desenvolvido}
\label{cha:trabalho desenvolvido}

A solução proposta para a resolução do problema apresentado em XXX é um sistema separado em três fases, cada uma delas ligada a um elemento físico: o Arduino, o telemóvel e a base de dados (alocada num computador). Assim surgem as seguintes secções, referentes às três fases do sistema.

\section{Funcionamendo do Arduino}
\label{sec:funcionamento_do_arduino}
Tal como descrito na secção XXX do capitulo XXX, o código Arduino está separado em três grupos chave específicos, sendo descrita a sua utilização específica na presente secção:

\begin{itemize}
\item Na zona de declarações são incluídas as bibliotecas referentes ao acelerómetro, ao cartão de memória e ao GPS.
São também definidos os modos de escrita no cartão de memória, os portos de ligação de comunicação do GPS e criadas variáveis globais referentes ao GPS e ao acelerómetro.
Além disso, são criadas variáveis globais de controlo do código, variáveis de passagem de parâmetros entre vários ciclos da função \emph{loop} e declaração de constantes que serão utilizadas diversas vezes ao longo do código, facilitando a alteração do seu valor em todo o código, caso tal se mostre necessário.

\item Na função \emph{setup} são declaradas as bandas de comunicação entre o Arduino e o telemóvel e entre o GPS e o Arduino.
São também inicializados dois \emph{leds} como elementos de saída de informação e um botão como entrada de informação e é atribuído o estado de “desligado” a ambos os \emph{leds}.
Por fim, são executadas as funções de inicialização do leitor do cartão de memória e do acelerómetro, SD\textunderscore init e AccelInit respectivamente, ambas explicadas mais á frente.

\begin{figure}[hbtp]
	\centering
	\includegraphics[height=10cm]{FlowArduino}
	\caption{Fluxograma referente ao código Arduino}
	\label{fig:Fluxograma_referente_ao_código_arduino}
\end{figure}

\item A função \emph{loop}, embora executada de uma forma constante, contém um código bastante simples mas importante.
A função começa por verificar se existe informação disponível para leitura no módulo Bluetooth.
Caso tal se verifique, essa informação é lida e comparada, para que seja possível determinar se a ligação ao módulo Bluetooth do Arduino sofreu alguma alteração, nomeadamente, quanto à ligação com a aplicação desenvolvida para esta dissertação.
Na hipótese desta alteração se verificar, é avaliada se a ligação foi estabelecida ou cortada, sendo que, no caso de ser estabelecida, é executada a função ReadSD para leitura do cartão de memória e um \emph{led} de controlo é aceso.
Caso contrário, o mesmo \emph{led} de controlo é desligado. 
De seguida são lidos os valores de aceleração dos eixos X, Y e Z, provenientes do acelerómetro, armazenados e comparados com a leitura anterior para verificação da existência de alguma irregularidade no asfalto.
A maneira como esta comparação será explicada mais à frente.
Caso seja detectada alguma irregularidade, um dos \emph{leds} de controlo pisca e são armazenados, numa variável \textbf{S}, os valores do nível da irregularidade, bem como o grupo data-hora proveniente do GPS.
Seguidamente é verificada a ligação à aplicação e caso esta exista, os valores de \textbf{S} são enviados directamente para o telemóvel.
Se a ligação não existir, os valores de \textbf{S} são transferidos para o cartão de memória para que possam ser transmitidos para o telemóvel assim que exista uma ligação estável com o mesmo.
Por fim, os valores actuais do acelerómetro são armazenados para que possam ser comparados na próxima execução da função \emph{loop}.
\end{itemize}

\section{Funções adicionais}
\label{sec:funcoes_adicionais}

\subsection{SD\textunderscore init}
\label{sub:sd_init}

Esta função serve para inicializar o leitor de cartões SD.
Nela é chamada a função \emph{begin} da biblioteca \emph{SD} e verificada a resposta dessa mesma função.
Caso a resposta seja o valor 10, um \emph{led} de controlo pisca para que o utilizador saiba que a inicialização ocorreu com sucesso e é verificada a existência de um ficheiro com informação previamente armazenada sobre irregularidades detetadas anteriormente.
Caso exista esse ficheiro, este é aberto de modo a que se possam juntar novos valores de irregularidades detetadas, caso contrário, o ficheiro é criado para futuros armazenamentos.

\subsection{WriteSD}
\label{sub:writesd}

Para que o cartão de memória possa armazenar dados é apenas necessário abrir um ficheiro em modo escrita.
A função \emph{open} da biblioteca do SD está desenhada de forma a que, quando é pedido para abrir um ficheiro, este seja procurado no cartão de memória e caso não exista nenhum ficheiro com esse nome, então é criado um novo ficheiro vazio, com o nome inserido e posteriormente aberto.
Para que possa ser mais fácil ao utilizador saber se a abertura e escrita do ficheiro foi feita com sucesso, um \emph{led} pisca duas vezes.
No final resta apenas fechar o ficheiro para que tudo fique guardado devidamente.

\subsection{ReadSD}
\label{sub:readsd}

Semelhante à função de escrita, a função de leitura tenta abrir um ficheiro com o nome pretendido embora, neste caso, se o ficheiro desejado não existir, não é criado um novo pois se este não existe significa que não existem dados pendentes para envio.
Quando um ficheiro é lido com sucesso, os seus valores são enviados pela porta série do Arduino para o telemóvel, o ficheiro é fechado e eliminado, de modo a que não sejam enviados dados duplicados.
O código está protegido de modo a que os dados sejam apenas enviados para um telemóvel com a aplicação desenvolvida pois previamente foi recebida uma mensagem de controlo enviada pela própria aplicação, informando o estado da ligação, evitando que os dados sejam enviados para ligações desconhecidas.

\subsection{AccelInit}
\label{sub:accelInit}

Nesta função o acelerómetro é ligado e são determinados os valores de sensibilidade do mesmo.
É também feita uma inicialização de valores para queda livre e batida que não são utilizados para esta dissertação.
Esta funcionalidade foi mantida para possíveis aplicações futuras e uma deteção mais pormenorizada das irregularidades.
Seguidamente, um \emph{led} de controlo pisca para o utilizador saber que a inicialização foi bem sucedida.

\subsection{Accel}
\label{sub:accel}

Na função \emph{Accel} é feito o processamento dos valores de aceleração dos três eixos do acelerómetro para determinar a classificação da irregularidade detetada.
Embora o processo seja simples, é eficaz e semelhante ao método \textbf{Z-DIFF} apresentado em XXX, mas aplicado aos três eixos de aceleração.
O método baseia-se em detetar a existência de desvios superiores a um limite previamente determinado, tendo sido considerados valores múltiplos de 100 nesta dissertação.
Assim, é feita uma subtração de valores sucessivos de aceleração dos eixos e avaliado o valor absoluto dessa diferença.
O valor atribuído à irregularidade é o mais baixo dos três desvios.

\subsection{displayGPS}
\label{sub:displayGPS}

A função \emph{displayGPS} é sem dúvida a mais elaborada nesta secção do trabalho devido aos dados enviados pelo GPS, muito extensos e, neste caso, não necessários, além de se encontrarem num formado não muito fácil de separar.
A função é chamada dentro da função \emph{loop} de modo a registar tanto o conjunto data-hora como as coordenadas em que a irregularidade foi detetada.
Quando é chamada, faz uma leitura dos valores que o sensor GPS recebe dos satélites e percorre-os até encontrar o texto "GPRMC", a partir do qual vem toda a informação necessária e armazena todos esses valores num vetor de strings (sequências de caracteres).
O passo seguinte consiste em percorrer este  e retirar apenas os segmentos relevantes, pela ordem desejada, neste caso, começando pela latitude e longitude, seguindo-se a data e por fim a hora.
Quando estes quatro parâmetros são armazenados, a função muda o valor de uma variável de controlo e o código prossegue.

\subsection{FTOA - Float to ASCII}
\label{sub:FTOA}

Dada a forma em que os valores das coordenadas do GPS são recebidos, é necessária fazer uma conversão para um formato mais simples de transmitir e como o processo de conversão é executado diversas vezes, foi criada uma função que evita repetição do código.
Os dados enviados pelo GPS, relativos às coordenadas vêm no formato DDS – Degree Decimal Minutes (Graus e Minutos Decimais em Português) e apresentam o formato (1) da figura \ref{fig:formatos_de_apresentacao_de_coordenadas}.
Tal como se pode verificar, existem seis parâmetros neste formato, nomeadamente graus, minutos  e hemisfério, três para a latitude e outros três para a longitude. 
Um formato mais simples é o DD – Decimal Degree (Graus Decimais em Português), também apresentado na figura \ref{fig:formatos_de_apresentacao_de_coordenadas}, no ponto (2).
Neste formato existem apenas quatro parâmetros (graus e hemisfério) que podem ser reduzidos a dois, se forem considerados graus negativos, representando o hemisfério a que a latitude ou longitude se referem, sendo assim necessário enviar menos dados para representar a mesma quantidade de informação, tornando o processo mais rápido.
A conversão de DDS para DD é simples, sendo apenas necessário dividir a parte dos minutos decimais por 60 (convertendo minutos para graus) e somar esse valor aos graus já existentes.
Como os valores que são recebidos do GPS vêm em strings, é necessário convertê-los para formato numérico, através da função atoi (ASCII to Integer, ASCCI para inteiro em português) que já existe nas bibliotecas do Arduino e posteriormente voltar a converter para formato de string.
Apesar de ser possível escrever um valor numérico numa string, quando esta operação é feita, apenas duas casas decimais são utilizadas e para que as coordenadas possam apresentar um elevado grau de precisão são necessárias quatro casas decimais, sendo assim necessária a criação desta nova função de modo a reter tantas casas decimais quanto as desejadas.

\begin{figure}[!htbp]
	\centering
	\includegraphics[height=2cm]{Coordenadas}
	\caption{Formatos de apresentação de coordenadas}
	\label{fig:formatos_de_apresentacao_de_coordenadas}
\end{figure}

\section{Funcionamento do Android}
\label{sec:funcionamento_do_android}

Tal como explicado na secção XXX do capítulo XXX, o método escolhido para a programação Android foi a MIT App Inventor 2, no website http://ai2.appinventor.mit.edu devido à facilidade apresentada, graças a ser uma programação por blocos bem como ao facto de correr num web browser e fazendo o processamento e compilação de código no servidor em vez de ser no cliente, diminuindo bastante o processamento feito no computador local, tornando o processo de criação mais rápido.

A base do código da aplicação é um ciclo while que corre desde que a aplicação esteja aberta, semelhante à função \emph{loop} do Arduino.
Dentro deste ciclo existem duas condições: uma que verifica se o GPS do telemóvel está a detetar coordenadas e outra que verifica a ligação ao módulo \emph{Bluetooth} do Arduino, bem como um procedimento que verifica a existência de dados armazenados no telemóvel para futuro envio para a base de dados.
O GPS do telemóvel é utilizado para que seja possível reportar irregularidades existentes no asfalto sem que se esteja a conduzir, possibilitando todos os transeunte adicionar informações sobre o local em que se encontram.
Sempre que são detetadas coordenadas disponíveis, um botão com o texto \emph{"Shock"} fica ativo e sempre que é pressionado são armazenadas as coordenadas atuais, bem como a data e hora presentes para um posterior envio para a base de dados.
Para que estas ocorrências possam ser diferenciadas das detetadas pelo acelerómetro montado no Arduino, o código que lhe é atribuído tem um valor único, facilitando assim a diferenciação entre os dois tipos de irregularidades.
No que toca à ligação a um módulo \emph{Bluetooth}, sempre que esta existe, é enviado um código para o dispositivo a que foi feita a ligação.
Este código serve para o Arduino saber que o dispositivo que se encontra ligado tem a aplicação a correr, evitando assim que os dados recolhidos por este sejam enviados para um dispositivo desconhecido.
Depois de ser feita a ligação, o código fica a "escutar" o Arduino até que ele envie alguma informação para o telemóvel e quando tal acontece, os vários conjuntos de dados  recebidos são divididos para uma lista temporária.
Após a divisão ser feita, a lista é percorrida percorrida e os dados são enviados para o cartão de memória do telemóvel ou para a base de dados, dependendo da existência de ligação Wi-Fi.
Esta ligação é verificada a partir de um erro produzido automaticamente pelo sistema Android, sendo emitida uma mensagem de erro com um código específico.
O envio de dados para a base de dados é feito através do método \emph{POST}, sendo desta forma possível ocultar os dados enviados, sendo esta uma vantagem no caso de no futuro serem desenvolvidas contas de utilizador em que certas informações devem permanecer confidenciais.
Se a ligação à \emph{internet} não existir, os dados são armazenados no telemóvel para um futuro envio, sendo necessário pressionar um botão existente na aplicação.
Este botão serve também para informar o utilizador que tem dados armazenados no dispositivo, uma vez que o botão nem sempre se encontra ativo.
No que toca à seleção de dispositivos \emph{Bluetooth}, é mostrada uma lista dos dispositivos com que o telemóvel está emparelhado.
No caso de ser a primeira vez que um determinado telemóvel se liga ao Arduino, é necessário fazer o emparelhamento fora da aplicação, como para qualquer outro dispositivo.
Quando a aplicação é terminada, é enviada uma mensagem para o Arduino de modo a que este saiba que não é possível enviar mais informação sobre irregularidades e armazene todas as deteções nesse cartão de memória.

\begin{figure}[hbtp]
	\centering
	\includegraphics[height=19cm]{FlowAndroid}
	\caption{Fluxograma referente ao código Android}
	\label{fig:Fluxograma_referente_ao_código_android}
\end{figure}

\newpage

\section{Base de dados}
\label{sec:Base_de_dados}

De modo a que seja possível consultar as irregularidades , foi necessário desenvolver uma base de dados onde ficassem armazenadas as coordenadas, data, hora e intensidade da irregularidade detectadas.
Esta base de dados foi desenvolvida em MySQL e contém cinco campos, sendo eles o ID (campo de identificação de cada entrada), a intensidade da irregularidade, a latitude e a longitude da irregularidade e também o conjunto data-hora, utilizando apenas uma variável.
Esta base de dados pode ainda ser consultada e alterada por um administrador da mesma, caso este ache necessário de modo a inserir novas funcionalidades no projeto.

\section{WebSite}
\label{sec:website}

\subsection{home.php}
\label{sub:home.php}

Esta é a página principal da parte \emph{web} desta dissertação.
Nela é possível ver-se o logótipo e o nome utilizados neste projeto e também um mapa que contém todas as irregularidades detetadas ou inseridas manualmente, referindo a intensidade das mesmas.
Para que essas irregularidades possam ser mostradas, é percorrida toda a base de dados através de uma \emph{query} (questão, em português) de MySQL em que são devolvidos os valores de latitude, longitude e intensidade.
Depois, utilizando uma pequena interface desenvolvida pela Google, todos os pontos são marcados no mapa, sendo que no seu sinalizador está indicada a intensidade da irregularidade, como se pode ver na figura \ref{fig:home.php}.
No caso de existirem várias irregularidades muito próximas fisicamente, a interface agrupa-as automaticamente, informando quantas se encontram nessa zona.
A cor destes agrupamentos é alterada consoante o número de itens aglomerados, tornando assim mais fácil a deteção de locais com elevado número de ocorrências.

\subsection{store.php}
\label{sub:store.php}

A página store não contém nenhuma componente gráfica para consulta de informação pois destina-se apenas ao armazenamento de dados na base de dados.
Quando o telemóvel envia dados, é este o destino e como tal, tem que existir algum processamento dos dados recebidos.
A informação recebida tem o aspecto da figura \ref{fig:informacao_tipo_adicionada_na_base_de_dados} e vem em quatro strings: força, latitude, longitude e data-hora, sendo portanto necessário fazer uma conversão para o formato correcto e consequente armazenamento em variáveis locais.
De seguida é feita uma query MySQL para que seja possível fazer o armazenamento na base de dados, sendo emitida uma mensagem de controlo para o telemóvel de modo a dar informação sobre o sucesso ou falha deste armazenamento.

\begin{figure}[htp]
	\centering
	\includegraphics[height=13cm]{home}
	\caption{Página \emph{web} home.php}
	\label{fig:home.php}
\end{figure}

\subsection{listLocations.php}
\label{sub:listlocations.php}

Esta página \emph{web} é bastante semelhante à \emph{home.php}, sendo que aqui apenas é mostrada a localização da irregularidade selecionada na página \emph{listlocations.php} de modo a ser mais fácil descobrir a localização da irregularidade quando esta se encontra agrupada com outras irregularidades existentes nas proximidades, como acontece na figura \ref{fig:home.php}.

\begin{figure}[htbp]
	\centering
	\includegraphics[width=15cm]{dados}
	\caption{informação tipo adicionada na base de dados}
	\label{fig:informacao_tipo_adicionada_na_base_de_dados}
\end{figure}

\subsection{showLocation.php}
\label{sub:showLocation.php}

Esta página tem a finalidade de consultar todas as irregularidades num formato numérico, ao invés da página Home.php, em que as irregularidades são apresentadas num mapa.
Tal como apresentado na figura \ref{fig:listlocations.php}, nesta página é possível consultar a toda a informação referente a uma irregularidade, incluindo a data e hora em que foi detetada, ao invés da página Home.php.
Para que seja necessário visualizar a lista de irregularidades, é necessário colocar no endereço o número da página que se deseja visualizar, sendo o valor pré-definido “1”.
Cada uma destas páginas mostra dez entradas da lista de irregularidades e no fundo da página é mostrado um navegador para as diferentes páginas. 
Alinhado com cada irregularidade, existe uma hiperligação para a página showLocation.php em que é mostrado no mapa a localização da irregularidade seleccionada para uma melhor compreensão dos valores de latitude e longitude.

\begin{figure}[htbp]
	\centering
	\includegraphics[width=15cm]{listlocations}
	\caption{Página \emph{web} listlocations.php}
	\label{fig:listlocations.php}
\end{figure}

\section{Montagem do sistema}
\label{sec:montagem_do_sistema}

O sistema final foi montado numa PCB compatível com o Arduino Mega, apesar de ter sido utilizado um Arduino Uno.
A utilização desta PCB deve-se ao tamanho da mesma, sendo assim possível espaçar os componentes mais facilmente, evitando interferências entre eles e que possibilita uma melhor gestão das ligações.
Foi também necessário utilizar dois reguladores de tensão para converter a tensão de entrada de 12V do automóvel para 5.0V e 3.3V.
Apesar do Arduino já conter reguladores semelhantes, existe uma limitação de 200mA na corrente de alimentação, a qual não é suficiente para alimentar o sistema.
São também necessários dois reguladores pois existem componentes que necessitam de ser alimentados a 5.0V e outros a 3.3V pelo que o regulador LM7805 faz a conversão de 12V para 5.0V e o regulador LM3940 faz a conversão de 5.0V para 3.3V.
As figuras \ref{fig:esquematico_do_sistema} e \ref{fig:montagem_real_do_sistema} representam um esquema da montagem e a montagem real do sistema, respectivamente.

\begin{figure}[htbp]
	\centering
	\includegraphics[width=15cm]{esquema}
	\caption{Esquemático do sistema}
	\label{fig:esquematico_do_sistema}
\end{figure}

\begin{figure}[htbp]
	\centering
	\includegraphics[width=15cm]{montagem}
	\caption{Montagem real do sistema}
	\label{fig:montagem_real_do_sistema}
\end{figure}

