\chapter{Conclusões}
\label{conclusoes}

Tal como esperado, o sistema que foi desenvolvido funciona e apresenta resultados bastante positivos.
O sistema foi capaz de detetar irregularidades existentes no asfalto e classificá-las consoante a sua intensidade, recorrendo ao acelerómetro.
Esta recolha de dados mostrou-se bastante precisa, graças ao local onde o acelerómetro foi instalado, fazendo com que o sistema possa ser instalado em diferentes veículos sem que seja necessário qualquer tipo de calibração, dado que as vibrações sentidas pelas rodas são diretamente transmitidas para o acelerómetro.
Outro fator importante é o custo do sistema, bastante reduzido e que pode ser ainda mais conveniente caso seja feita a sua construção em série.
Como o sistema é compacto e tem um tamanho reduzido, a sua instalação é muito fácil e rápida na maioria dos casos e não causa qualquer tipo de alteração visual no veículo.
Apesar da aplicação não se encontrar disponível nas principais lojas do género, a sua instalação é bastante fácil graças à possibilidade de transferência pelo browser do telemóvel.
A sua utilização é também bastante simples, com um número reduzido de botões que estão devidamente identificados quanto à sua função.
No que toca às páginas \emph{web}, a página de consulta de irregularidades é bastante simples, para uma mais fácil navegação entre irregularidades, sendo bastante explicita e acessível a todos os utilizadores.
A página principal do sistema mostra a informação essencial das deteções num mapa, sendo muito intuitiva a sua utilização, bem como a possibilidade de concluir quais as áreas que afetam mais utilizadores.

\section{Trabalho futuro}
\label{sec:trabalho_futuro}

Existem sempre alguns detalhes a melhorar nos sistemas existentes e este não falha à regra.
Apesar de detetar irregularidades, a avaliação das irregularidade detetadas não permite a diferenciação entre uma depressão no asfalto ou uma protuberância.
Ao nível da aplicação, existem algumas falhas no armazenamento, maioritariamente devido a falhas na comunicação Bluetooth, uma vez que não são reconhecidos pacotes enviados mas não recebidos.
O armazenamento de dados no telemóvel, embora não ocupe muita memória, pode sempre ser otimizado, alterando o tipo de variáveis a guardar, bem como fazendo a sua compressão.
No que toca às páginas \emph{web}, são bastante funcionais e práticas mas não mostram grande apelo visual, sendo apenas uma ferramenta de consulta dos dados registados.
Idealmente será também interessante que seja feita uma parceria com uma empresa que tenha uma grande frota de veículos, uma vez que a validação de dados e o seu grau de relevância se torna mais facilmente detetável se vários utilizadores fizerem a deteção de irregularidades.
Esta ideia poderá ser promovida junto de entidades responsáveis pela gestão das vias, através de incentivos aos seus utilizadores.