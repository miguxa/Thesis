\chapter{Resumo}
\label{cha:ressumo}

O trabalho descrito nesta dissertação foi desenvolvido de maneira a poder ajudar a solucionar o problema que são as estradas irregulares.
A metodologia proposta é apenas uma possibilidade entre muitas outras funcionais que tenta estabelecer uma ligação com a "Internet das Coisas" e com as "Cidades Inteligentes", dois conceitos que têm apresentado um grande crescimento nos últimos anos.

O trabalho desenvolvido está dividido em três partes, cada uma diretamente relacionada com um dispositivo específico.
A primeira parte, em que é feita a recolha de dados, utiliza um microprocessador Arduino Uno, que guarda e analisa valores de recolhidos por um acelerómetro e os junta a informação fornecida por um recetor GPS.
Através de uma ligação Bluetooth, esta informação é passada para um telemóvel que representa a segunda fase do processo, onde os dados são armazenados para um posterior envio para o último passo do processo.
Nesse último passo, os dados são recebidos num servidor, via Wi-Fi e são dispostos num \emph{web site} para uma fácil consulta dos utilizadores.

Todo este processo permitiu o estudo e utilização de várias tecnologias, bem como a interação entre as mesmas, tornando-se assim num projeto bastante diversificado com focos em diferentes áreas.