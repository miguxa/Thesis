The work described in this dissertation was developed in order to help solve the problem of irregular roads.
The proposed methodology is just one possibility among many other that also work and tries to establish a connection with the ``Internet of Things'' and with ``Intelligent Cities'', two concepts that have shown a great growth in recent years.
The developed work is divided into three parts, each one directly related to a specific device.
The first part, in which the data is collected, uses an Arduino Uno microprocessor, which stores and analyzes values collected by an accelerometer and joins the information provided by a GPS receiver.
Through a Bluetooth connection, this information is passed to a mobile phone that represents the second phase of the process, where the data is stored for a subsequent send to the last step of the process.
In this last step, the data is received on a server via Wi-Fi and is arranged in a web site for easy user usage.
All this process allowed the study and use of several technologies, as well as the interaction between them, thus becoming a very diversified project with focuses in different areas.