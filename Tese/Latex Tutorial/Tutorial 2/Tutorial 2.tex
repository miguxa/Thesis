\documentclass[11pt]{article}
\usepackage[utf8]{inputenc}
%como escrever índices superiores e inferiores

\begin{document}

índices em cima:
$$2x^3$$
$$2x^{34}$$
$$2x^34$$
%ver a diferença entre ter chavetas ou não
$$2x^{3x+4}$$
$$2x^{3x^4+5}$$

índices em baixo:
$$x_1$$
$$x_{12}$$
$${x_1}_{12}$$

itálico:
$$\pi$$
$$\alpha$$
$$\gamma$$
$$A=\pi \times r^2$$

funções trigonométricas:
$$y=\sin (x)$$
$$y=\tan (x)$$
$$y=\cos (x)$$
$$y=\log (x)$$
$$y=\ln (x)$$
$$y=\sqrt{2}$$
$$y=\sqrt[3]{8}$$

frações:

Cerca de $\frac{2}{3}$ do copo está cheio.

Isto é texto $\displaystyle{\frac{2}{5}}$ que continua aqui.
%o comando acima torna a fração do tamanho do texto

$$\frac{x}{x^2+x+1}$$
$$\frac{\sqrt[2]{x+1}}{x+2}$$
$$\frac{\frac{1}{1+x}}{2-3x}$$
\end{document}