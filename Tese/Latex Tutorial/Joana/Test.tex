\documentclass[11pt]{article}
\usepackage[utf8]{inputenc}
\usepackage{multicol}
\usepackage{amsmath}
\setlength{\columnsep}{2cm}
%Este é o pacote dos acentos.
%Muito necessário ter!!
\begin{document}
\begin{center}
{\huge Trabalho de férias}
\end{center}
\begin{center}
{\huge para Joana Carvalho}
\end{center}
\vspace{10mm}

{\large 1- Mostra que:}
% Enter entre linhas separa-as
\begin{multicols}{2}
a) $ \frac {(3 - \sqrt{12})^{2} }  {(2 + 3 \sqrt{3} )}= \frac{87 \sqrt{3} - 150}{23}$

b) $ \frac {(1 + \sqrt{27})^{2} }  {(1 + 5 \sqrt{3} )}= -\frac{59 + 73 \sqrt{3}}{37}$\\
\end{multicols}
\vspace{5mm}

{\large 2- Determina J e M de modo a que obtenhas as raízes assinaladas e descobre a restante.}\\

a) $x^{3}+Jx^{2} + Mx+56$, raízes: 2 e 4\\

b) $x^3-7x^2-Jx-M$, raízes: 5 e -1\\
\vspace{5mm}

{\large 3- Determina as 3 raízes dos seguintes polinómios:}
\begin{multicols}{3}
a) $x^3-x$

b) $x^3-x^2$

c) $x^3-2x^2+x$
\end{multicols}
\vspace{5mm}

{\large 4- Resolve em ordem a $x$}
\begin{multicols}{2}
a) $-2<5x+3<8$

b) $2x-\frac{4-x}{3}\leq5x$
\end{multicols}
\vspace{5mm}

{\large 5- Representa em potência de base 5}

\vspace{5mm}
\begin{center} 
{\large $\frac{5^{\frac{1}{6}}\times5^{\frac{1}{2}}}{(\sqrt[3]{5})^{-2}}$}
\end{center}

\break
{\large 1-}

\vspace{5mm}
a)\hspace{5mm} $ \frac {(3 - \sqrt{12})^{2} }  {(2 + 3 \sqrt{3} )}=$
   $ \frac{(3-\sqrt{12})(3-\sqrt{12})}{(2+3\sqrt{3})}=$
   $ \frac{(9-6\sqrt{12}+12)}{(2+3\sqrt{3})}=$
   $ \frac{(21-6\sqrt{12})}{(2+3\sqrt{3})}=$
   $ \frac{(21-6\sqrt{12})(2-3\sqrt{3})}{(2+3\sqrt{3})(2-3\sqrt{3})}=$
   $= \frac{42-63\sqrt{3}-12\sqrt{12}+18\sqrt{36}}{4-9\times3}=$
   $ \frac{42-63\sqrt{3}-12\sqrt{4\times3}+18\times6}{4-27}=$
   $ \frac{42+108-63\sqrt{3}-12\times2\sqrt{3}}{-23}=$
   $= \frac{150-87\sqrt{3}}{-23}=$
   $\frac{87 \sqrt{3} - 150}{23}$ 

\vspace{5mm}
b)\hspace{5mm} $\frac{(1+\sqrt{27})^2}{(1+5\sqrt{3})}=$
	$ \frac{1+2\sqrt{27}+27}{(1+5\sqrt{3})}=$
	$ \frac{(28+2\sqrt{27})}{(1+5\sqrt{3})}=$
	$ \frac{(28+2\sqrt{27})(1-5\sqrt{3})}{(1+5\sqrt{3})(1-5\sqrt{3})}=$
	$ \frac{28-140\sqrt{3}+2\sqrt{27}-10\sqrt{81}}{1-25\times3}=$
	$= \frac{28-140\sqrt{3}+2\sqrt{9\times3}-10\times9}{1-75}=$
	$ \frac{28-90-140\sqrt{3}+2\times3\sqrt{3}}{-74}=$
	$ \frac{-62-134\sqrt{3}}{-74}=$
	$ \frac{31+67\sqrt{3}}{37}$

\vspace{5mm}
{\large 2-}

a)

\begin{table}[htbp]
\centering
\label{my-label}
\begin{tabular}{c|cccc}
  & 1 & J                        & M                           & 56         \\
2 &   & 2                        & 4+2J                        & 8+4J+2M    \\ \hline
  & 1 & 2+J                      & \multicolumn{1}{c|}{4+2J+M} & 64+4J+2M=0 \\ \cline{5-5} 
4 &   & 4                        & 24+4J                       &            \\ \cline{1-4}
  & 1 & \multicolumn{1}{c|}{6+J} & 28+6J+M=0                   &            \\ \cline{4-4}
\end{tabular}
\end{table}

\begin{minipage}{0.3\textwidth}
	\[\left\{
		\begin{array}{lr}
    		64+4J+2M=0\\
    		28+6J+M=0
  		\end{array}
	\right.
	\]
\end{minipage}%
\begin{minipage}{0.3\textwidth}
	\[(=)\left\{
		\begin{array}{lr}
    		-------\\
    		M=-28-6J
  		\end{array}
	\right.
	\]					
\end{minipage}
\begin{minipage}{0.3\textwidth}
	\[(=)\left\{
		\begin{array}{lr}
    		64+4J-56-12J=0\\
    		-----------
  		\end{array}
	\right.
	\]			
\end{minipage}

\begin{minipage}{0.3\textwidth}
	\[(=)\left\{
		\begin{array}{lr}
    		8=8J\\
    		----
  		\end{array}
	\right.
	\]
\end{minipage}%
\begin{minipage}{0.3\textwidth}
	\[(=)\left\{
		\begin{array}{lr}
    		J=1\\
    		28+6+M=0
  		\end{array}
	\right.
	\]					
\end{minipage}
\begin{minipage}{0.3\textwidth}
	\[(=)\left\{
		\begin{array}{lr}
    		J=1\\
    		M=-34
  		\end{array}
	\right.
	\]			
\end{minipage}

\vspace{5mm}
Substituindo J no ultimo resultado do Ruffini vem:
$x + (6 +1) = 0$ \hspace{1mm} $(=)$ \hspace{1mm} $x = -7$
e obtém-se a ultima raiz.
\vspace{5mm}

b)
\begin{table}[htbp]
\centering
\label{my-label}
\begin{tabular}{c|cccc}
   & 1 & -7                      & -J                         & -M         \\
5  &   & 5                       & -10                        & -5J-50     \\ \hline
   & 1 & -2                      & \multicolumn{1}{c|}{-J-10} & -M-5J-50=0 \\ \cline{5-5} 
-1 &   & -1                      & 4                          &            \\ \cline{1-4}
   & 1 & \multicolumn{1}{c|}{-3} & -J-7=0                     &            \\ \cline{4-4}
\end{tabular}
\end{table}
\break

\begin{minipage}{0.3\textwidth}
	\[\left\{
		\begin{array}{lr}
    		-M-5J-50=0\\
    		-J-7=0
  		\end{array}
	\right.
	\]
\end{minipage}%
\begin{minipage}{0.3\textwidth}
	\[(=)\left\{
		\begin{array}{lr}
    		-------\\
    		J=-7
  		\end{array}
	\right.
	\]					
\end{minipage}
\begin{minipage}{0.3\textwidth}
	\[(=)\left\{
		\begin{array}{lr}
    		-M+35-50\\
    		-----------
  		\end{array}
	\right.
	\]			
\end{minipage}

\begin{minipage}{0.3\textwidth}
	\[(=)\left\{
		\begin{array}{lr}
    		-M-15=0\\
    		------
  		\end{array}
	\right.
	\]
\end{minipage}%
\begin{minipage}{0.3\textwidth}
	\[(=)\left\{
		\begin{array}{lr}
    		M=-15\\
    		J=-7
  		\end{array}
	\right.
	\]					
\end{minipage}

\vspace{5mm}
A partir do último resultado do Ruffini, aparece:
$x-3=0$ \hspace{1mm} $(=)$ \hspace{1mm} $x=3$\\

\vspace{5mm}

{\large 3-}

a) $x^3-x=0 (=) x(x^2-1)=0 (=) x=0 \lor x^2-1=0 (=) x=0 \lor x^2=1 (=) \break  (=) x=0 \lor x=\pm\sqrt{1} (=) x=0 \lor x=-1 \lor x=1$
\vspace{5mm}

b) $x^3-x^2=0 (=) x\times x\times (x-1) =0 (=) x=0 \lor x=0 \lor x-1 =0 (=)\break (=) x=0 \lor x=0 \lor x=1$
\vspace{5mm}

c) $x^3-2x^2+x = 0(=) x(x^2-2x+1)=0 (=) x=0\lor x^2-2x+1=0 (=)\break (=) x=0 \lor x=1 \lor x=1$
\vspace{5mm}

{\large 4-}

a) $-2<5x+3<8 (=) -5<5x<5 (=) -1<x<1$
\vspace{5mm}

b) $2x-\frac{4-x}{3}\leq 5x (=)6x-4+x\leq 15x (=) 6x-15x+x \leq 4 (=) -8x \leq 4 (=)\break (=) 8x \geq -4 (=) x\geq-\frac{1}{2}$\\
\vspace{5mm}

{\large 5-}
\vspace{5mm}

$\frac{5^\frac{1}{6}\times 5^\frac{1}{2}}{(\sqrt[3]{5})^{-2}} = 
\frac{5^\frac{4}{6}}{5^{\frac{-2}{3}}}=
5^{(\frac{4}{6}-\frac{-2}{3})}=
5^{(\frac{4}{6}+\frac{4}{6})}=
5^{\frac{8}{6}}=
5^{\frac{4}{3}}$
\end{document}
