\documentclass[11pt]{article}
\usepackage[utf8]{inputenc}
\usepackage{multicol}
\setlength{\columnsep}{2cm}
%Este é o pacote dos acentos.
%Muito necessário ter!!
\begin{document}
\begin{center}
{\huge Trabalho de férias}
\end{center}
\begin{center}
{\huge para Joana Carvalho}
\end{center}
\vspace{10mm}

{\large 1- Mostra que:}
% Enter entre linhas separa-as
\begin{multicols}{2}
a) $ \frac {(3 - \sqrt{12})^{2} }  {(2 + 3 \sqrt{3} )}= \frac{87 \sqrt{3} - 150}{23}$

b) $ \frac {(1 + \sqrt{27})^{2} }  {(1 + 5 \sqrt{3} )}= -\frac{59 + 73 \sqrt{3}}{37}$\\
\end{multicols}
\vspace{5mm}

{\large 2- Determina J e M de modo a que obtenhas as raízes assinaladas e descobre a restante.}\\

a) $x^{3}+Jx^{2} + Mx+56$, raízes: 2 e 4\\

b) $x^3-7x^2-Jx-M$, raízes: 5 e -1\\
\vspace{5mm}

{\large 3- Determina as 3 raízes dos seguintes polinómios:}
\begin{multicols}{3}
a) $x^3-x$

b) $x^3-x^2$

c) $x^3-2x^2+x$
\end{multicols}
\vspace{5mm}

{\large 4- Resolve em ordem a $x$}
\begin{multicols}{2}
a) $-2<5x+3<8$

b) $2x-\frac{4-x}{3}<=5x$
\end{multicols}
\vspace{5mm}

{\large 5- Representa em potência de base 5}

\vspace{5mm}
\begin{center} 
{\large $\frac{5^{\frac{1}{6}}\times5^{\frac{1}{2}}}{(\sqrt[3]{5})^{-2}}$}
\end{center}
\end{document}